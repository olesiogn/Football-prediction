%\thispagestyle{empty}
\section{CONCLUSÃO}
	Conclui-se que em questão de dados para os modelos é mais difícil prever o resultado de Casa x Fora x Empate e Vitória x Empate x Derrota. Os modelos performam melhor quando o alvo da previsão é binário (2 classes). Dentre as bases com 2 classes Fora x Empate e Derrota x Empate foram as que tiveram menor performance e Casa x Fora e Vitória x Derrota foram as que tiveram melhor performance. Podemos concluir então que os modelos performam melhor em prever resultados sem empate.
	
	Levando em consideração os métodos de divisão das bases de dados concluimos que o método de validação cruzada gera uma performance melhor do que o método de divisão 70\% treino 30\% teste.
	
	A tabela abaixa mostra quais classificadores obtiveram uma melhor performance em cada método de divisão e em cada abordagem.

\begin{table}[h]
\resizebox{\textwidth}{!}{%
\begin{tabular}{|l|l|l|l|} 
\hline
\multicolumn{4}{|c|}{\textbf{70-30}} \\ \hline
 & \textbf{Premier League} & \textbf{Serie A} & \textbf{Brasileirão} \\ \hline
\textbf{CxFxE} & Regressão Logística & Regressão Logística & MLP \\ \hline
\textbf{CxF} & Regressão Logística & Regressão Logística & MLP  \\ \hline
\textbf{CxE} & Floresta Aleatória & Floresta Aleatória & KNN \\ \hline
\textbf{FxE} & Regressão Logística & KNN & Árvore de Decisão \\ \hline
\multicolumn{4}{|c|}{\textbf{Validação Cruzada}} \\ \hline
\textbf{CxFxE} & KNN & SVM & Regressão Logística \\ \hline
\textbf{CxF} & MLP & MLP & MLP  \\ \hline
\textbf{CxE} & SVM & SVM & KNN \\ \hline
\textbf{FxE} & SVM & SVM & Árvore de Decisão/Floresta Aleatória \\ \hline
\end{tabular}%
}
\caption{Classificadores com melhor performance em cada método de divisão e abordagem.}
\label{cron}
\end{table}

	Podemos concluir que os algoritmos que obtiveram a melhor performance mais vezes foram nesta ordem: regressão logística, SVM e MLP, KNN, floresta aleatória e árvore de decisão.
	
	Todos os algoritmos abordados neste trabalho foram implementados em Python e encontram-se dísponíveis em um repositório público do GitHub.